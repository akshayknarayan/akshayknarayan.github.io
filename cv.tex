% This is the default prefix for Scribble-generated Latex
\documentclass{article}

\usepackage[utf8]{inputenc}
\usepackage[T1]{fontenc}

\newcommand{\packageGraphicx}{\usepackage{graphicx}}
\newcommand{\packageHyperref}{\usepackage{hyperref}}
\newcommand{\renewrmdefault}{\renewcommand{\rmdefault}{ptm}}
\newcommand{\packageRelsize}{\usepackage{relsize}}
% amsmath is required for the combination of {mathabx,
% wasysym, newtxmath} to work. Otherwise, newtxmath
% would load amsmath *after* mathabx and wasysym,
% causing command redefinition issues.
\newcommand{\packageAmsmath}{\usepackage{amsmath}}
\newcommand{\packageMathabx}{\usepackage{mathabx}}
% Avoid conflicts between "mathabx" and "wasysym",
% and between "wasysym" integrals and "amsmath" integrals (iint).
\newcommand{\packageWasysym}{
  \let\leftmoon\relax \let\rightmoon\relax \let\fullmoon\relax \let\newmoon\relax \let\diameter\relax
  \usepackage[nointegrals]{wasysym}}
% Both newtxmath and mathabx define the \widering command.
% The only reason we choose the newtxmath version is that
% acmart.cls is also using the one from newtxmath.
\newcommand{\packageTxfonts}{
  \let\widering\relax
  \let\oldwidebar\widebar
  \let\widebar\relax
  \usepackage{newtxmath}
  % if newtxmath is before version 1.7.0,
  % then we are still going to use widebar from mathabx
  \ifx\widebar\relax
    \let\widebar\oldwidebar
  \fi
}
\newcommand{\packageTextcomp}{\usepackage{textcomp}}
\newcommand{\packageFramed}{\usepackage{framed}}
\newcommand{\packageHyphenat}{\usepackage[htt]{hyphenat}}
\newcommand{\packageColor}{\usepackage[usenames,dvipsnames]{color}}
\newcommand{\doHypersetup}{\hypersetup{bookmarks=true,bookmarksopen=true,bookmarksnumbered=true}}
\newcommand{\packageTocstyle}{}
\newcommand{\packageCJK}{\IfFileExists{CJK.sty}{\usepackage{CJK}}{}}
% this bit intentionally left blank
% This is the default style configuration for Scribble-generated Latex

\packageGraphicx
\packageHyperref
\renewrmdefault
\packageRelsize
\packageAmsmath
\packageMathabx
\packageWasysym
\packageTxfonts
\packageTextcomp
\packageFramed
\packageHyphenat
\packageColor
\doHypersetup
\packageTocstyle
\packageCJK


%%%%%%%%%%%%%%%%%%%%%%%%%%%%%%%%%%%%%%%%%%%%%%%%%%%%%%%%%%%%%%%%%%%%%%%%%%%%%%%%
% Configuration that is especially meant to be overridden:

% Inserted before every ``chapter'', useful for starting each one on a new page:
\newcommand{\sectionNewpage}{}
% Inserted before every book ``part''
\newcommand{\partNewpage}{\sectionNewpage}

% Hooks for actions within the `document' environment:
\newcommand{\preDoc}{}
\newcommand{\postDoc}{}

% Generated by `secref'; first arg is section number, second is section title:
\newcommand{\BookRef}[2]{\emph{#2}}
\newcommand{\ChapRef}[2]{\SecRef{#1}{#2}}
\newcommand{\SecRef}[2]{section~#1}
\newcommand{\PartRef}[2]{part~#1}
% Generated by `Secref':
\newcommand{\BookRefUC}[2]{\BookRef{#1}{#2}}
\newcommand{\ChapRefUC}[2]{\SecRefUC{#1}{#2}}
\newcommand{\SecRefUC}[2]{Section~#1}
\newcommand{\PartRefUC}[2]{Part~#1}

% Variants of the above with a label for an internal reference:
\newcommand{\BookRefLocal}[3]{\hyperref[#1]{\BookRef{#2}{#3}}}
\newcommand{\ChapRefLocal}[3]{\hyperref[#1]{\ChapRef{#2}{#3}}}
\newcommand{\SecRefLocal}[3]{\hyperref[#1]{\SecRef{#2}{#3}}}
\newcommand{\PartRefLocal}[3]{\hyperref[#1]{\PartRef{#2}{#3}}}
\newcommand{\BookRefLocalUC}[3]{\hyperref[#1]{\BookRefUC{#2}{#3}}}
\newcommand{\ChapRefLocalUC}[3]{\hyperref[#1]{\ChapRefUC{#2}{#3}}}
\newcommand{\SecRefLocalUC}[3]{\hyperref[#1]{\SecRefUC{#2}{#3}}}
\newcommand{\PartRefLocalUC}[3]{\hyperref[#1]{\PartRefUC{#2}{#3}}}

% Variants of the above with a section number is empty (i.e., UnNumbered):
\newcommand{\BookRefUN}[1]{\BookRef{}{#1}}
\newcommand{\ChapRefUN}[1]{\SecRefUN{#1}}
\newcommand{\SecRefUN}[1]{``#1''}
\newcommand{\PartRefUN}[1]{\SecRefUN{#1}}
\newcommand{\BookRefUCUN}[1]{\BookRefUN{#1}}
\newcommand{\ChapRefUCUN}[1]{\ChapRefUN{#1}}
\newcommand{\SecRefUCUN}[1]{\SecRefUN{#1}}
\newcommand{\PartRefUCUN}[1]{\PartRefUN{#1}}

\newcommand{\BookRefLocalUN}[2]{\hyperref[#1]{\BookRefUN{#2}}}
\newcommand{\ChapRefLocalUN}[2]{\SecRefLocalUN{#1}{#2}}
\newcommand{\SecRefLocalUN}[2]{\hyperref[#1]{\SecRefUN{#2}}}
\newcommand{\PartRefLocalUN}[2]{\SecRefLocalUN{#1}{#2}}
\newcommand{\BookRefLocalUCUN}[2]{\BookRefLocalUN{#1}{#2}}
\newcommand{\ChapRefLocalUCUN}[2]{\ChapRefLocalUN{#1}{#2}}
\newcommand{\SecRefLocalUCUN}[2]{\SecRefLocalUN{#1}{#2}}
\newcommand{\PartRefLocalUCUN}[2]{\PartRefLocalUN{#1}{#2}}

\newcommand{\SectionNumberLink}[2]{\hyperref[#1]{#2}}

% Enabled with a 'enable-index-merge part style property. This default
% implementation isn't good enough, because the argument is a
% comma-separated sequence of labels:
\newcommand{\Smanypageref}[1]{\pageref{#1}}

%%%%%%%%%%%%%%%%%%%%%%%%%%%%%%%%%%%%%%%%%%%%%%%%%%%%%%%%%%%%%%%%%%%%%%%%%%%%%%%%
% Fonts

% Font commands used by generated text:
\newcommand{\Scribtexttt}[1]{{\texttt{#1}}}
\newcommand{\textsub}[1]{$_{\hbox{\textsmaller{#1}}}$}
\newcommand{\textsuper}[1]{$^{\hbox{\textsmaller{#1}}}$}
\newcommand{\intextcolor}[2]{\textcolor{#1}{#2}}
\newcommand{\intextrgbcolor}[2]{\textcolor[rgb]{#1}{#2}}
\newcommand{\incolorbox}[2]{{\fboxrule=0pt\fboxsep=0pt\protect\colorbox{#1}{#2}}}
\newcommand{\inrgbcolorbox}[2]{{\fboxrule=0pt\fboxsep=0pt\protect\colorbox[rgb]{#1}{#2}}}
\newcommand{\plainlink}[1]{#1}
\newcommand{\techoutside}[1]{#1}
\newcommand{\techinside}[1]{#1}
\newcommand{\badlink}[1]{#1}
\newcommand{\indexlink}[1]{#1}
\newcommand{\noborder}[1]{#1}
\newcommand{\Smaller}[1]{\textsmaller{#1}}
\newcommand{\Larger}[1]{\textlarger{#1}}
\newcommand{\planetName}[1]{PLane\hspace{-0.1ex}T}
\newcommand{\slant}[1]{{\textsl{#1}}}

% Used for <, >, and | in tt mode. For some fonts and installations,
% there seems to be an encoding issue, so pick T1 explicitly:
\newcommand{\Stttextmore}{{\fontencoding{T1}\selectfont>}}
\newcommand{\Stttextless}{{\fontencoding{T1}\selectfont<}}
\newcommand{\Stttextbar}{{\fontencoding{T1}\selectfont|}}

%%%%%%%%%%%%%%%%%%%%%%%%%%%%%%%%%%%%%%%%%%%%%%%%%%%%%%%%%%%%%%%%%%%%%%%%%%%%%%%%
% Tables

% The `stabular' environment seems to be the lesser of evils among 
%  page-breaking table environments (and we've made a copy as ``pltstabular'
%  to make sure that it doesn't change).

\makeatletter
%%%%%%%%%%%%%%%%%%%%%%%%%%%%%%%%%%%%%%%%%%%%%%%%%%%%%%%%%%%%%%%%%%%%%%
\message{pltstabular is a modification of stabular}
%% A renamed version of:
%% stabular.sty
%% Copyright 1998 Sigitas Tolu\v sis
%% VTeX Ltd., Akademijos 4, Vilnius, Lithuania
%% e-mail sigitas@vtex.lt
%% http://www.vtex.lt/tex/download/macros/
%%
% This program can redistributed and/or modified under the terms
% of the LaTeX Project Public License Distributed from CTAN
% archives in directory macros/latex/base/lppl.txt; either
% version 1 of the License, or (at your option) any later version.
%
% PURPOSE:   Improve tabular environment.
%
% SHORT DESCRIPTION:
%
% Changed internal commands: \@mkpream, \@addamp, \@xhline
%
% Provides new commands in tabular (used after command \\):
% \emptyrow[#1] 
% -------------
%    Adds empty row, #1 - height of the row 
%
% \tabrow{#1}[#2] 
% ---------------
%    Adds row of natural height: #1\\[#2]
%
% Provides new environments: pltstabular and pltstabular* 
%                            --------     ---------
%            One more multi-page version of tabular
%
%
\def\empty@finalstrut#1{%
  \unskip\ifhmode\nobreak\fi\vrule\@width\z@\@height\z@\@depth\z@}
\def\no@strut{\global\setbox\@arstrutbox\hbox{%
    \vrule \@height\z@
           \@depth\z@
           \@width\z@}%
    \gdef\@endpbox{\empty@finalstrut\@arstrutbox\par\egroup\hfil}%
}%
\def\yes@strut{\global\setbox\@arstrutbox\hbox{%
    \vrule \@height\arraystretch \ht\strutbox
           \@depth\arraystretch \dp\strutbox
           \@width\z@}%
    \gdef\@endpbox{\@finalstrut\@arstrutbox\par\egroup\hfil}%
}%
\def\@mkpream#1{\@firstamptrue\@lastchclass6
  \let\@preamble\@empty\def\empty@preamble{\add@ins}%
  \let\protect\@unexpandable@protect
  \let\@sharp\relax\let\add@ins\relax
  \let\@startpbox\relax\let\@endpbox\relax
  \@expast{#1}%
  \expandafter\@tfor \expandafter
    \@nextchar \expandafter:\expandafter=\reserved@a\do
       {\@testpach\@nextchar
    \ifcase \@chclass \@classz \or \@classi \or \@classii \or \@classiii
      \or \@classiv \or\@classv \fi\@lastchclass\@chclass}%
  \ifcase \@lastchclass \@acol
      \or \or \@preamerr \@ne\or \@preamerr \tw@\or \or \@acol \fi}
\def\@addamp{%
  \if@firstamp
    \@firstampfalse
    \edef\empty@preamble{\add@ins}%
  \else
    \edef\@preamble{\@preamble &}%
    \edef\empty@preamble{\expandafter\noexpand\empty@preamble &\add@ins}%
  \fi}
\newif\iftw@hlines \tw@hlinesfalse
\def\@xhline{\ifx\reserved@a\hline
               \tw@hlinestrue
             \else\ifx\reserved@a\Hline
               \tw@hlinestrue
             \else
               \tw@hlinesfalse
             \fi\fi
      \iftw@hlines
        \aftergroup\do@after
      \fi
      \ifnum0=`{\fi}%
}
\def\do@after{\emptyrow[\the\doublerulesep]}
\def\emptyrow{\noalign\bgroup\@ifnextchar[\@emptyrow{\@emptyrow[\z@]}}
\def\@emptyrow[#1]{\no@strut\gdef\add@ins{\vrule \@height\z@ \@depth#1 \@width\z@}\egroup%
\empty@preamble\\
\noalign{\yes@strut\gdef\add@ins{\vrule \@height\z@ \@depth\z@ \@width\z@}}%
}
\def\tabrow#1{\noalign\bgroup\@ifnextchar[{\@tabrow{#1}}{\@tabrow{#1}[]}}
\def\@tabrow#1[#2]{\no@strut\egroup#1\ifx.#2.\\\else\\[#2]\fi\noalign{\yes@strut}}
%
\def\endpltstabular{\crcr\egroup\egroup \egroup}
\expandafter \let \csname endpltstabular*\endcsname = \endpltstabular
\def\pltstabular{\let\@halignto\@empty\@pltstabular}
\@namedef{pltstabular*}#1{\def\@halignto{to#1}\@pltstabular}
\def\@pltstabular{\leavevmode \bgroup \let\@acol\@tabacol
   \let\@classz\@tabclassz
   \let\@classiv\@tabclassiv \let\\\@tabularcr\@stabarray}
\def\@stabarray{\m@th\@ifnextchar[\@sarray{\@sarray[c]}}
\def\@sarray[#1]#2{%
  \bgroup
  \setbox\@arstrutbox\hbox{%
    \vrule \@height\arraystretch\ht\strutbox
           \@depth\arraystretch \dp\strutbox
           \@width\z@}%
  \@mkpream{#2}%
  \edef\@preamble{%
    \ialign \noexpand\@halignto
      \bgroup \@arstrut \@preamble \tabskip\z@skip \cr}%
  \let\@startpbox\@@startpbox \let\@endpbox\@@endpbox
  \let\tabularnewline\\%
%    \let\par\@empty
    \let\@sharp##%
    \set@typeset@protect
    \lineskip\z@skip\baselineskip\z@skip
    \@preamble}

%%%%%%%%%%%%%%%%%%%%%%%%%%%%%%%%%%%%%%%%%%%%%%%%%%%%%%%%%%%%%%%%%%%%%%
\makeatother

\newenvironment{bigtabular}{\begin{pltstabular}}{\end{pltstabular}}
% For the 'boxed table style:
\newcommand{\SBoxedLeft}{\textcolor[rgb]{0.6,0.6,1.0}{\vrule width 3pt\hspace{3pt}}}
% Formerly used to keep the horizontal line for a definition on the same page:
\newcommand{\SEndFirstHead}[0]{ \nopagebreak \\ }
% Corrects weirdness when a table is the first thing in
%  an itemization:
\newcommand{\bigtableinlinecorrect}[0]{~

\vspace{-\baselineskip}\vspace{\parskip}}
% Used to indent the table correctly in an itemization, since that's
%  one of the things stabular gets wrong:
\newlength{\stabLeft}
\newcommand{\bigtableleftpad}{\hspace{\stabLeft}}
\newcommand{\atItemizeStart}[0]{\addtolength{\stabLeft}{\labelsep}
                                \addtolength{\stabLeft}{\labelwidth}}


% For a single-column table in simple environments, it's better to
%  use the `list' environment instead of `stabular'.
\newenvironment{SingleColumn}{\begin{list}{}{\topsep=0pt\partopsep=0pt%
\listparindent=0pt\itemindent=0pt\labelwidth=0pt\leftmargin=0pt\rightmargin=0pt%
\itemsep=0pt\parsep=0pt}\item}{\end{list}}

%%%%%%%%%%%%%%%%%%%%%%%%%%%%%%%%%%%%%%%%%%%%%%%%%%%%%%%%%%%%%%%%%%%%%%%%%%%%%%%%
% Etc.

% ._ and .__
\newcommand{\Sendabbrev}[1]{#1\@}
\newcommand{\Sendsentence}[1]{\@#1}

% Default style for a nested flow:
\newenvironment{Subflow}{\begin{list}{}{\topsep=0pt\partopsep=0pt%
\listparindent=0pt\itemindent=0pt\labelwidth=0pt\leftmargin=0pt\rightmargin=0pt%
\itemsep=0pt}\item}{\end{list}}

% For the 'inset nested-flow style:
\newenvironment{SInsetFlow}{\begin{quote}}{\end{quote}}

% Indent a 'code-inset nested flow:
\newcommand{\SCodePreSkip}{\vskip\abovedisplayskip}
\newcommand{\SCodePostSkip}{\vskip\belowdisplayskip}
\newenvironment{SCodeFlow}{\SCodePreSkip\begin{list}{}{\topsep=0pt\partopsep=0pt%
\listparindent=0pt\itemindent=0pt\labelwidth=0pt\leftmargin=2ex\rightmargin=2ex%
\itemsep=0pt\parsep=0pt}\item}{\end{list}\SCodePostSkip}
\newcommand{\SCodeInsetBox}[1]{\setbox1=\hbox{\hbox{\hspace{2ex}#1\hspace{2ex}}}\vbox{\SCodePreSkip\vtop{\box1\SCodePostSkip}}}

% Inset a 'vertical-inset nested flow:
\newcommand{\SVInsetPreSkip}{\vskip\abovedisplayskip}
\newcommand{\SVInsetPostSkip}{\vskip\belowdisplayskip}
\newenvironment{SVInsetFlow}{\SVInsetPreSkip\begin{list}{}{\topsep=0pt\partopsep=0pt%
\listparindent=0pt\itemindent=0pt\labelwidth=0pt\leftmargin=0pt\rightmargin=0pt%
\itemsep=0pt\parsep=0pt}\item}{\end{list}\SVInsetPostSkip}
\newcommand{\SVInsetBox}[1]{\setbox1=\hbox{\hbox{#1}}\vbox{\SCodePreSkip\vtop{\box1\SCodePostSkip}}}

% The 'compact itemization style:
\newenvironment{compact}{\begin{itemize}}{\end{itemize}}
\newcommand{\compactItem}[1]{\item #1}

% The nested-flow style for `centerline':
\newenvironment{SCentered}{\begin{trivlist}\item \centering}{\end{trivlist}}

% The \refpara command corresponds to `margin-note'. The
% refcolumn and refcontent environments also wrap the note,
% because they simplify the CSS side.
\newcommand{\refpara}[1]{\normalmarginpar\marginpar{\raggedright \footnotesize #1}}
\newcommand{\refelem}[1]{\refpara{#1}}
\newenvironment{refcolumn}{}{}
\newenvironment{refcontent}{}{}

\newcommand{\refparaleft}[1]{\reversemarginpar\marginpar{\raggedright \footnotesize #1}}
\newcommand{\refelemleft}[1]{\refparaleft{#1}}
\newenvironment{refcolumnleft}{}{}

% Macros used by `title' and `author':
\newcommand{\titleAndVersionAndAuthors}[3]{\title{#1\\{\normalsize \SVersionBefore{}#2}}\author{#3}\maketitle}
\newcommand{\titleAndVersionAndEmptyAuthors}[3]{\title{#1\\{\normalsize \SVersionBefore{}#2}}#3\maketitle}
\newcommand{\titleAndEmptyVersionAndAuthors}[3]{\title{#1}\author{#3}\maketitle}
\newcommand{\titleAndEmptyVersionAndEmptyAuthors}[3]{\title{#1}\maketitle}
\newcommand{\titleAndVersionAndAuthorsAndShort}[4]{\title[#4]{#1\\{\normalsize \SVersionBefore{}#2}}\author{#3}\maketitle}
\newcommand{\titleAndVersionAndEmptyAuthorsAndShort}[4]{\title[#4]{#1\\{\normalsize \SVersionBefore{}#2}}#3\maketitle}
\newcommand{\titleAndEmptyVersionAndAuthorsAndShort}[4]{\title[#4]{#1}\author{#3}\maketitle}
\newcommand{\titleAndEmptyVersionAndEmptyAuthorsAndShort}[4]{\title[#4]{#1}\maketitle}
\newcommand{\SAuthor}[1]{#1}
\newcommand{\SAuthorSep}[1]{\qquad}
\newcommand{\SVersionBefore}[1]{Version }

% Useful for some styles, such as sigalternate:
\newcommand{\SNumberOfAuthors}[1]{}

\let\SOriginalthesubsection\thesubsection
\let\SOriginalthesubsubsection\thesubsubsection

% sections
\newcommand{\Spart}[2]{\part[#1]{#2}}
\newcommand{\Ssection}[2]{\section[#1]{#2}\let\thesubsection\SOriginalthesubsection}
\newcommand{\Ssubsection}[2]{\subsection[#1]{#2}\let\thesubsubsection\SOriginalthesubsubsection}
\newcommand{\Ssubsubsection}[2]{\subsubsection[#1]{#2}}
\newcommand{\Ssubsubsubsection}[2]{{\bf #2}}
\newcommand{\Ssubsubsubsubsection}[2]{\Ssubsubsubsection{#1}{#2}}

% "star" means unnumbered and not in ToC:
\newcommand{\Spartstar}[1]{\part*{#1}}
\newcommand{\Ssectionstar}[1]{\section*{#1}\renewcommand*\thesubsection{\arabic{subsection}}\setcounter{subsection}{0}}
\newcommand{\Ssubsectionstar}[1]{\subsection*{#1}\renewcommand*\thesubsubsection{\arabic{section}.\arabic{subsubsection}}\setcounter{subsubsection}{0}}
\newcommand{\Ssubsubsectionstar}[1]{\subsubsection*{#1}}
\newcommand{\Ssubsubsubsectionstar}[1]{{\bf #1}}
\newcommand{\Ssubsubsubsubsectionstar}[1]{\Ssubsubsubsectionstar{#1}}

% "starx" means unnumbered but in ToC:
\newcommand{\Spartstarx}[2]{\Spartstar{#2}\phantomsection\addcontentsline{toc}{part}{#1}}
\newcommand{\Ssectionstarx}[2]{\Ssectionstar{#2}\phantomsection\addcontentsline{toc}{section}{#1}}
\newcommand{\Ssubsectionstarx}[2]{\Ssubsectionstar{#2}\phantomsection\addcontentsline{toc}{subsection}{#1}}
\newcommand{\Ssubsubsectionstarx}[2]{\Ssubsubsectionstar{#2}\phantomsection\addcontentsline{toc}{subsubsection}{#1}}
\newcommand{\Ssubsubsubsectionstarx}[2]{\Ssubsubsubsectionstar{#2}}
\newcommand{\Ssubsubsubsubsectionstarx}[2]{\Ssubsubsubsubsectionstar{#2}}

% "grouper" is for the 'grouper style variant --- on subsections and lower,
%  because \Spart is used for grouper at the section level. Grouper implies
%  unnumbered.
\newcounter{GrouperTemp}
\newcommand{\Ssubsectiongrouper}[2]{\setcounter{GrouperTemp}{\value{subsection}}\Ssubsectionstarx{#1}{#2}\setcounter{subsection}{\value{GrouperTemp}}}
\newcommand{\Ssubsubsectiongrouper}[2]{\setcounter{GrouperTemp}{\value{subsubsection}}\Ssubsubsectionstarx{#1}{#2}\setcounter{subsubsection}{\value{GrouperTemp}}}
\newcommand{\Ssubsubsubsectiongrouper}[2]{\Ssubsubsubsectionstarx{#1}{#2}}
\newcommand{\Ssubsubsubsubsectiongrouper}[2]{\Ssubsubsubsubsectionstarx{#1}{#2}}

\newcommand{\Ssubsectiongrouperstar}[1]{\setcounter{GrouperTemp}{\value{subsection}}\Ssubsectionstar{#1}\setcounter{subsection}{\value{GrouperTemp}}}
\newcommand{\Ssubsubsectiongrouperstar}[1]{\setcounter{GrouperTemp}{\value{subsubsection}}\Ssubsubsectionstar{#1}\setcounter{subsubsection}{\value{GrouperTemp}}}
\newcommand{\Ssubsubsubsectiongrouperstar}[1]{\Ssubsubsubsectionstar{#1}}
\newcommand{\Ssubsubsubsubsectiongrouperstar}[1]{\Ssubsubsubsubsectionstar{#1}}

\newcommand{\Ssubsectiongrouperstarx}[2]{\setcounter{GrouperTemp}{\value{subsection}}\Ssubsectionstarx{#1}{#2}\setcounter{subsection}{\value{GrouperTemp}}}
\newcommand{\Ssubsubsectiongrouperstarx}[2]{\setcounter{GrouperTemp}{\value{subsubsection}}\Ssubsubsectionstarx{#1}{#2}\setcounter{subsubsection}{\value{GrouperTemp}}}
\newcommand{\Ssubsubsubsectiongrouperstarx}[2]{\Ssubsubsubsectionstarx{#1}{#2}}
\newcommand{\Ssubsubsubsubsectiongrouperstarx}[2]{\Ssubsubsubsubsectionstarx{#1}{#2}}

% Generated by `subsubsub*section':
\newcommand{\SSubSubSubSection}[1]{\Ssubsubsubsubsectionstar{#1}}

% For hidden parts with an empty title:
\newcommand{\notitlesection}{\vspace{2ex}\phantomsection\noindent}

% To increment section numbers:
\newcommand{\Sincpart}{\stepcounter{part}}
\newcommand{\Sincsection}{\stepcounter{section}}
\newcommand{\Sincsubsection}{\stepcounter{subsection}}
\newcommand{\Sincsubsubsection}{\stepcounter{subsubsection}}
\newcommand{\Sincsubsubsubsection}{}
\newcommand{\Sincsubsubsubsubsection}{}

% When brackets appear in section titles:
\newcommand{\SOpenSq}{[}
\newcommand{\SCloseSq}{]}

% Helper for box-mode macros:
\newcommand{\Svcenter}[1]{$\vcenter{#1}$}

% Verbatim
\newenvironment{SVerbatim}{}{}

% Helper to work around a problem with "#"s for URLs within \href
% within other macros:
\newcommand{\Shref}[3]{\href{#1\##2}{#3}}

% For URLs:
\newcommand{\Snolinkurl}[1]{\nolinkurl{#1}}

% History note:
\newcommand{\SHistory}[1]{\begin{smaller}#1\end{smaller}}

%%%%%%%%%%%%%%%%%%%%%%%%%%%%%%%%%%%%%%%%%%%%%%%%%%%%%%%%%%%%%%%%%%%%%%%%%%%%%%%%

% Scribble then generates the following:
%
%  \begin{document}
%  \preDoc
%  \titleAndVersion{...}{...}
%  ... document content ...
%  \postDoc
%  \end{document}
\usepackage{fullpage,color,xspace,enumitem}
\usepackage[compact]{titlesec}
%\usepackage[type1=true]{libertine}
\titleformat{\section}{\Large\scshape}{}{0em}{}[\titlerule]
\titleformat{\subsection}{\bf}{}{0em}{}[]
%\PassOptionsToPackage{hyphens}{url}
\hypersetup{
  colorlinks=true,
  linkcolor=blue,
  urlcolor=blue,
}
%\urlstyle{same}
\usepackage{changepage}

\newcommand{\identity}[1]{#1}

%\usepackage[hyphens,spaces,obeyspaces]{url}
\renewcommand{\packageHyperref}{\usepackage[colorlinks=true,citecolor=magenta]{hyperref}}
\begin{document}
\preDoc
\date{}
\titleAndEmptyVersionAndEmptyAuthors{Akshay Narayan}{}{}
\label{t:x28part_x22Akshayx5fNarayanx22x29}

\vspace{-1.8cm}
\noindent
\begin{tabular*}{\textwidth}{@{}l@{\extracolsep{\fill}}r@{}}
115 Waterman St., CIT 545, Providence, RI 02906 & {akshayn@brown.edu $\cdot$ \url{https://akshayn.xyz}} \\
\end{tabular*}

\section{Education}
\noindent
\begin{tabular*}{\textwidth}{@{\hspace{1cm}}l@{\extracolsep{\fill}}r@{}}
\textbf{Massachusetts Institute of Technology}. & \emph{2016 --- 2022} \\
M.S. Computer Science, 2019 & \\
Ph.D. Computer Science, 2022 & \\
Thesis Title: \emph{\href{https://dspace.mit.edu/handle/1721.1/144577}{Enabling Configurable, Extensible, and Modular Network Stacks}} & \\
Advisors: Hari Balakrishnan, Mohammad Alizadeh. & \\
& \\
\textbf{University of California, Berkeley}. & \emph{2011 --- 2015} \\ 
Bachelor of Science in Electrical Engineering and Computer Science with High Honors, 2015. & \\
\end{tabular*}

\section{Professional Appointments}
\begin{tabular*}{\textwidth}{@{\hspace{1cm}}l@{\extracolsep{\fill}}r@{}}
\textbf{Brown University}, Providence, RI, USA  &\\
Assistant Professor, Computer Science. & \emph{July 2024 ---} \\
%
\noindent \textbf{University of California, Berkeley}, Berkeley, CA, USA &\\
Postdoctoral Research Fellow, NetSys Lab. & \emph{August 2022 --- June 2024} \\
Staff Researcher, NetSys Lab, advised by Sylvia Ratnasamy and Scott Shenker. & \emph{Spring 2014 --- August 2016}\\
%
\textbf{Microsoft}, Redmond, WA, USA &\\
Research Intern in Mobility and Networking, advised by Behnaz Arzani. & \emph{May 2018 --- August 2018} \\
Software Development Engineer Intern. Operating Systems Group. & \emph{May 2014 --- August 2014}\\
%
\textbf{Jive Software}, Palo Alto, CA, USA &\\
Engineering Intern. & \emph{May 2013 --- August 2013}\\
%
\textbf{NASA Ames Research Center}, Mountain View, CA, USA &\\
Education Associates Program. & \emph{May 2012 --- November 2012}\\
\end{tabular*}

\section{Honors and Awards}
\begin{tabular*}{\textwidth}{@{\hspace{1cm}}r@{\hspace{5pt}}l@{\extracolsep{\fill}}r@{}}
\textbf{ACM EuroSys}     & Best Artifact Award              & \emph{2021} \\
\textbf{ACM SIGCOMM}     & Best Student Paper Award         & \emph{2018} \\
\textbf{National Science Foundation}         & NSF Graduate Research Fellowship & \emph{2017} \\
\textbf{Massachusetts Institute of Technology}         & Jacobs Presidential Fellowship   & \emph{2016} \\
\textbf{University of California, Berkeley} & Boeing Scholar                   & \emph{2014} \\
\textbf{University of California, Berkeley} & Eta Kappa Nu                     & \emph{2012} \\
\textbf{University of California, Berkeley} & Regents and Chancellors Scholar  & \emph{2011} \\
\end{tabular*}


\identity{\section{Publications}}

\identity{
\smallskip
\begin{adjustwidth}{1cm}{}
{\footnotesize
Conferences are the primary publication venue in Computer Science.
By convention, author order lists students and post-docs first (ordered by contribution), followed by faculty in no fixed order. Some publications list faculty authors in alphabetical order or order of contribution, others list the lead faculty author last.
}
\end{adjustwidth}
}

\identity{\begin{itemize}[leftmargin=1.5cm]}
\identity{\item}Franco Solleza, Justus Adam, \identity{\textbf{Akshay Narayan}}, Malte Schwarzkopf, Andrew Crotty, Nesime Tatbul\identity{\\}{``}Kernel Extension DSLs Should Be Verifier{-}Safe!{''}\identity{\\}\identity{\emph{eBPF}}\identity{ }\identity{2025}\identity{\item}Rahul Bothra, Venkat Arun, Brighten Godfrey, \identity{\textbf{Akshay Narayan}}, Ahmed Saeed\identity{\\}{``}Lightweight Automated Reasoning for Network Architectures{''}\identity{\\}\identity{\emph{HotNets}}\identity{ }\identity{2024}\identity{\item}Margarida Ferreira, Ranysha Ware, Yash Kothari, In\^{e}s Lynce, Ruben Martins, \identity{\textbf{Akshay Narayan}}, Justine Sherry\identity{\\}{``}Reverse{-}Engineering Congestion Control Algorithm Behavior{''}\identity{\\}\identity{\emph{IMC}}\identity{ }\identity{2024}\identity{\item}Lloyd Brown, Albert Gran Alcoz, Frank Cangialosi, \identity{\textbf{Akshay Narayan}}, Mohammad Alizadeh, Hari Balakrishnan, Eric Friedman, Ethan Katz{-}Bassett, Arvind Krishnamurthy, Michael Schapira, Scott Shenker\identity{\\}{``}Principles for Internet Congestion Management{''}\identity{\\}\identity{\emph{SIGCOMM}}\identity{ }\identity{2024}\identity{\item}Lloyd Brown, Yash Kothari, \identity{\textbf{Akshay Narayan}}, Arvind Krishnamurthy, Aurojit Panda, Justine Sherry, Scott Shenker\identity{\\}{``}How I Learned to Stop Worrying About CCA Contention{''}\identity{\\}\identity{\emph{Hotnets}}\identity{ }\identity{2023}\identity{\item}William Sussman, Emily Marx, Venkat Arun, \identity{\textbf{Akshay Narayan}}, Mohammad Alizadeh, Hari Balakrishnan, Aurojit Panda, Scott Shenker\identity{\\}{``}The Case for an Internet Primitive for Fault Localization{''}\identity{\\}\identity{\emph{Hotnets}}\identity{ }\identity{2022}\identity{\item}Prateesh Goyal, \identity{\textbf{Akshay Narayan}}, Frank Cangialosi, Srinivas Narayana, Mohammad Alizadeh, Hari Balakrishnan\identity{\\}{``}Elasticity Detection: A Building Block for Internet Congestion Control{''}\identity{\\}\identity{\emph{SIGCOMM}}\identity{ }\identity{2022}\identity{\item}Margarida Ferreira, \identity{\textbf{Akshay Narayan}}, In\^{e}s Lynce, Ruben Martins, Justine Sherry\identity{\\}{``}Counterfeiting Congestion Control Algorithms{''}\identity{\\}\identity{\emph{Hotnets}}\identity{ }\identity{2021}\identity{\item}Frank Cangialosi, \identity{\textbf{Akshay Narayan}}, Prateesh Goyal, Radhika Mittal, Mohammad Alizadeh, Hari Balakrishnan\identity{\\}{``}Site{-}to{-}Site Internet Traffic Control{''}\identity{\\}\identity{\emph{EuroSys}}\identity{ }\identity{2021}\identity{\item}\identity{\textbf{Akshay Narayan}}, Aurojit Panda, Mohammad Alizadeh, Hari Balakrishnan, Arvind Krishnamurthy, Scott Shenker\identity{\\}{``}Bertha: Tunneling through the Network API{''}\identity{\\}\identity{\emph{Hotnets}}\identity{ }\identity{2020}\identity{\item}Hongzi Mao, Parimarjan Negi, \identity{\textbf{Akshay Narayan}}, Hanrui Wang, Jiacheng Yang, Haonan Wang, Ryan Marcus, Ravichandra Addanki, Mehrdad Khani Shirkoohi, Songtao He, Vikram Nathan, Frank Cangialosi, Shaileshh Venkatakrishnan, Wei{-}Hung Weng, Song Han, Tim Kraska, Mohammad Alizadeh\identity{\\}{``}Park: An Open Platform for Learning{-}Augmented Computer Systems{''}\identity{\\}\identity{\emph{NeurIPS}}\identity{ }\identity{2019}\identity{\item}\identity{\textbf{Akshay Narayan}}, Frank Cangialosi, Deepti Raghavan, Prateesh Goyal, Srinivas Narayana, Radhika Mittal, Mohammad Alizadeh, Hari Balakrishnan\identity{\\}{``}Restructuring Endpoint Congestion Control{''}\identity{\\}\identity{\emph{SIGCOMM}}\identity{ }\identity{2018}\identity{\item}Saksham Agarwal, Shijin Rajakrishnan, \identity{\textbf{Akshay Narayan}}, Rachit Agarwal, David Shmoys, Amin Vahdat\identity{\\}{``}Sincronia: Near{-}Optimal Network Design for Coflows{''}\identity{\\}\identity{\emph{SIGCOMM}}\identity{ }\identity{2018}\identity{\item}\identity{\textbf{Akshay Narayan}}, Frank Cangialosi, Prateesh Goyal, Srinivas Narayana, Mohammad Alizadeh, Hari Balakrishnan\identity{\\}{``}The Case for Moving Congestion Control Out of the Datapath{''}\identity{\\}\identity{\emph{HotNets}}\identity{ }\identity{2017}\identity{\item}Peter Gao, \identity{\textbf{Akshay Narayan}}, Sagar Karandikar, Joao Carreira, Sangjin Han, Rachit Agarwal, Sylvia Ratnasamy, Scott Shenker\identity{\\}{``}Network Requirements for Resource Disaggregation{''}\identity{\\}\identity{\emph{OSDI}}\identity{ }\identity{2016}\identity{\item}Peter Gao, \identity{\textbf{Akshay Narayan}}, Gautam Kumar, Rachit Agarwal, Sylvia Ratnasamy, Scott Shenker\identity{\\}{``}pHost: Distributed Near{-}Optimal Datacenter Transport Over Commodity Network Fabric.{''}\identity{\\}\identity{\emph{CoNEXT}}\identity{ }\identity{2015}
\identity{\end{itemize}}

\section{Teaching}

\smallskip
\begin{adjustwidth}{1cm}{}
{\footnotesize Below, ``Instructor effectiveness rating'' is the percentage of students who answered either ``Agree'' or ``Strongly Agree'' to the statement ``Overall, I rate this instructor as effective.'' in an end-of-semester survey.}
\end{adjustwidth}
\medskip

\noindent
\begin{tabular*}{\textwidth}{@{\hspace{1cm}}l@{\extracolsep{\fill}}r@{}}
\textbf{Designing High-Performance Network Systems (CSCI 1675)}, Brown University. & \emph{Spring 2025, Fall 2025} \\
\end{tabular*}
\begin{itemize}[leftmargin=1.5cm]
\setlength{\itemsep}{0pt}
  \item Newly-developed advanced systems course.
  \item Spring 2025: 21 students; Instructor effectiveness rating: 92\%
\end{itemize}

\noindent
\begin{tabular*}{\textwidth}{@{\hspace{1cm}}l@{\extracolsep{\fill}}r@{}}
\textbf{Computer Networks and the Internet (CSCI 2680)}, Brown University. & \emph{Fall 2024} \\
\end{tabular*}
\begin{itemize}[leftmargin=1.5cm]
\setlength{\itemsep}{0pt}
  \item Newly-developed graduate seminar course.
  \item Fall 2025: 5 students;  Instructor effectiveness rating: 100\%
\end{itemize}


\identity{\section{Advising}}

\identity{\noindent\hspace{1cm}}
\identity{\textbf{M.Sc. research advisor}}, Brown University:

\identity{\begin{itemize}[leftmargin=1.5cm]\setlength{\itemsep}{0pt}}\identity{\item}\identity{\begin{tabular*}{\dimexpr \textwidth -1.5cm}[t]{@{}l@{\extracolsep{\fill}}r@{}}}Ziyun "Alice" Song\identity{& }\identity{\emph{October 2024}}\identity{ --- }\identity{\\}\identity{\end{tabular*}}\identity{\end{itemize}}

\identity{\noindent\hspace{1cm}}
\identity{\textbf{Undergraduate research advisor}}, Brown University:

\identity{\begin{itemize}[leftmargin=1.5cm]\setlength{\itemsep}{0pt}}\identity{\item}\identity{\begin{tabular*}{\dimexpr \textwidth -1.5cm}[t]{@{}l@{\extracolsep{\fill}}r@{}}}Hailey Hsiung\identity{& }\identity{\emph{September 2025}}\identity{ --- }\identity{\\}\identity{\end{tabular*}}\identity{\item}\identity{\begin{tabular*}{\dimexpr \textwidth -1.5cm}[t]{@{}l@{\extracolsep{\fill}}r@{}}}Hee Su "Julie" Chung\identity{& }\identity{\emph{September 2025}}\identity{ --- }\identity{\\}\identity{\end{tabular*}}\identity{\item}\identity{\begin{tabular*}{\dimexpr \textwidth -1.5cm}[t]{@{}l@{\extracolsep{\fill}}r@{}}}Bhavani Venkatesan\identity{& }\identity{\emph{May 2025}}\identity{ --- }\identity{\\}\identity{\end{tabular*}}\identity{\item}\identity{\begin{tabular*}{\dimexpr \textwidth -1.5cm}[t]{@{}l@{\extracolsep{\fill}}r@{}}}Edward Nguyen\identity{& }\identity{\emph{February 2025}}\identity{ --- }\identity{\\}\identity{\end{tabular*}}\identity{\item}\identity{\begin{tabular*}{\dimexpr \textwidth -1.5cm}[t]{@{}l@{\extracolsep{\fill}}r@{}}}Megan Zheng\identity{& }\identity{\emph{January 2025}}\identity{ --- }\identity{\\}\identity{\end{tabular*}}\identity{\item}\identity{\begin{tabular*}{\dimexpr \textwidth -1.5cm}[t]{@{}l@{\extracolsep{\fill}}r@{}}}Mengistie Hailemariam\identity{& }\identity{\emph{November 2024}}\identity{ --- }\identity{\\}\identity{\end{tabular*}}\identity{\end{itemize}}

\identity{\noindent\hspace{1cm}}
\identity{\textbf{Research advisor}}, Massachusetts Institute of Technology:

\identity{\begin{itemize}[leftmargin=1.5cm]\setlength{\itemsep}{0pt}}\identity{\item}\identity{\begin{tabular*}{\dimexpr \textwidth -1.5cm}[t]{@{}l@{\extracolsep{\fill}}r@{}}}Aditi Srinivasan\identity{ $\rightarrow$ Jane Street Capital}\identity{& }\identity{\emph{October 2018}}\identity{ --- }\identity{\emph{December 2020}}\identity{\\}\identity{Undergraduate and M.Eng. student & \\}\identity{Thesis: \href{https://dspace.mit.edu/handle/1721.1/130715}{Measuring and Optimizing for Network Conditions on Drones} & \\}\identity{\end{tabular*}}\identity{\end{itemize}}

\identity{\section{Professional Service}}

\identity{\noindent\hspace{1cm}
\textbf{University and Departmental Service}
\begin{itemize}[leftmargin=1.5cm]\setlength{\itemsep}{0pt}
  \item Brown University, Advanced Undergraduate Research Fellowships Review Committee: 2025
  \item Brown University, Dept. of Computer Science, UTRA Faculty Mentor: Summer 2025, Fall 2025
  \item Brown University, Dept. of Computer Science, Brown Systems Week Faculty Mentor: \href{brown-systems-week.github.io}{Summer 2025}
  \item Brown University, Dept. of Computer Science, PhD Admissions Committee: 2024, 2025
  \item Brown University, Dept. of Computer Science, exploreCSR Faculty Mentor: \href{https://explorecsr.cs.brown.edu/systems/index.html}{Fall 2024}
\end{itemize}
}

\identity{
\smallskip
\begin{adjustwidth}{1cm}{}
{\footnotesize
Conferences are the primary publication venue in Computer Science. Service to the profession centers on conference program committee memberships.
}
\end{adjustwidth}
}

\identity{\noindent\hspace{1cm}}\identity{\textbf{Conference and Workshop Program Committees}}\identity{\begin{itemize}[leftmargin=1.5cm]\setlength{\itemsep}{0pt}}\identity{\item }NSDI\identity{: }\identity{\href{https://www.usenix.org/conference/nsdi26/call-for-papers}{2026}}, \identity{\href{https://www.usenix.org/conference/nsdi24/call-for-papers}{2024}}\identity{\item }eBPF\identity{: }\identity{\href{https://conferences.sigcomm.org/sigcomm/2025/workshop/ebpf/}{2025}}\identity{\item }SIGCOMM\identity{: }\identity{\href{https://conferences.sigcomm.org/sigcomm/2025/tpc/}{2025}}\identity{\item }CoNEXT\identity{: }\identity{\href{https://conferences.sigcomm.org/co-next/2025/\#!/pc}{2025}}\identity{\item }SysDW\identity{: }\identity{\href{https://sysdw24.github.io/}{2024}}\identity{\item }HotNets\identity{: }\identity{\href{https://conferences.sigcomm.org/hotnets/2024/program-committee/}{2024}}\identity{\item }ANRW\identity{: }\identity{\href{https://www.irtf.org/anrw/2024/committees.html}{2024}}, \identity{\href{https://www.irtf.org/anrw/2023/committees.html}{2023}}\identity{\item }APNet\identity{: }\identity{\href{https://conferences.sigcomm.org/events/apnet2024/pc.php}{2024}}\identity{\end{itemize}}\identity{\noindent\hspace{1cm}}\identity{\textbf{Journal Reviewer}}\identity{\begin{itemize}[leftmargin=1.5cm]\setlength{\itemsep}{0pt}}\identity{\item }SIGCOMM CCR\identity{: }\identity{\href{https://ccronline.sigcomm.org/editorial-board/}{2025}}, \identity{\href{https://ccronline.sigcomm.org/editorial-board/}{2024}}\identity{\end{itemize}}\identity{\noindent\hspace{1cm}}\identity{\textbf{Artifact Evaluation Committee}}\identity{\begin{itemize}[leftmargin=1.5cm]\setlength{\itemsep}{0pt}}\identity{\item }SIGCOMM\identity{: }\identity{\href{https://conferences.sigcomm.org/sigcomm/2021/cf-artifacts.html}{2021}}\identity{\end{itemize}}

\postDoc
\end{document}
